% !TEX root = ../main.tex
Modern data collection is increasingly focused on recording data continuously over various domains of interest, including time or space.  
Many times, this collected data are more appropriately viewed as realizations of a function rather than finite-dimensional vectors.  
This has led to the development and rise of functional data analysis (FDA).  
FDA extends classical statistical models to handle infinitely dimensional data.  
FDA represents each observational unit, be it a growth curve, temperature history, or kinematic motion, as a smooth function defined on an interval.  
Modeling these data as functions has several attractive characteristics, including allowing for the ability to capture nonlinear behaviors naturally, but it also presents several challenges.  
These challenges include how to represent an infinitely dimensional object finitely, a tradeoff between smoothness and sparsity, as well as how to handle covariance structures.
There are many problems that appear when this topic is presented.  
One of these is functional regression, in particular, function on scalar regression (FoS).  In FoS models, the response, or dependent variable, is a function rather than a scalar.  
The goal of FoS models is to capture the relationship between entire functions.  
Some examples of this may be the modeling of a kinematic trajectory (presumed a curve) conditional on various covariates or modeling growth curves in response to environmental factors.  

My work builds on existing foundations to estimate coefficient curves in FoS regression.  
In Section 2, I will present a partial history of statistics, detailing significant developments and how they laid the foundation for functional data analysis.  
In Section 3, I provide some background on Bayesian methods specifically.  
Section 4 presents a specific existing model that serves as a building block for future methods.  
Section 5 presents the multivariate model, allowing for the estimation of multivariate functional outcomes.   